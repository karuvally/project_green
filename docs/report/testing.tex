\chapter{Testing and Implementation}
\section{Testing methods done for the project}
\par
System testing is the stage of implementation which is aimed at ensuring that
the system works accurately and efficiently before live operation commences.
Testing is the process of executing the program with the intent of finding
errors and missing operations and also complete verification to determine
whether the objective are met and the user requirements are satisfied.\\

The ultimate aim is quality assurance. Tests are carried and the results are
compared with the expected document. In that case of erroneous results,debugging
is done. Using detailed testing strategies a test plan is carried out on each
module. The test plan defines the unit,integration and system testing approach.
The test scope includes the following: A primary objective of testing
application systems is to assure that the system meets the full functional
requirements, including quality requirements(Non functional requirements).\\

At the end of the project development cycle, the user should find that the
project has met or exceeded all of their expectations as detailed in
requirements. Any changes, additions or deletions to the requirements document,
functional specification or design specification will be documented and tested
at the highest level of quality allowed within the remaining time of the
iproject and within the ability of the test team.\\

The secondary objective of testing application systems will be do:identify and
expose all issues and associated risks, communicate all known issues are
addressed in an appropriate matter before release. This test approach document
describes the appropriate strategies, process, work flows and methodologies
used to plan, organize, execute and manage testing of software project
"Real-time Traffic Congestion Analyzer for Road Safety"

\newpage

\section{Unit Testing}
\begin{table}[ht]

\resizebox{\textwidth}{!}{\begin{tabular}{|c| p{2cm} |p{3.7cm} |p{2.8cm}|c|}
\hline

Sl No & Procedures & Expected result & Actual result & Pass or Fail\\  
\hline

1 & Generate signature & Signature generated & Same as expected & Pass\\ 
\hline

2 & Verify signature & Signature verified & Same as expected & Pass\\
\hline

3 & Encrypt message & Message encrypted & Same as expected & Pass\\
\hline

4 & Decrypt message & message decrypted & Same as expected & Pass\\
\hline

5 & Read configuration & Configuration read & Same as expected & pass \\
\hline
 
\end{tabular}}
\caption{Unit test cases and results}
\end{table}

\section{Integration Testing}
\begin{table}[ht]

\resizebox{\textwidth}{!}{\begin{tabular}{|c| p{2cm} |p{3.7cm} |p{2.8cm}|c|}
\hline

Sl No & Procedures & Expected result & Actual result & Pass or Fail\\  
\hline

1 & Client-Server pairing & Pairing works  & Same as expected & Pass\\ 
\hline

2 & Web server start & Web server started & Same as expected & Pass\\
\hline

3 & Server send commands & Commands sent & Same as expected & Pass\\
\hline

3 & Client receives commands & Command received & Same as expected & Pass\\
\hline

\end{tabular}}
\caption{Unit test cases and results}
\end{table}

\newpage

\section{System Testing}
\begin{table}[ht]

\resizebox{\textwidth}{!}{\begin{tabular}{|c| p{2cm} |p{3.7cm} |p{2.8cm}|c|}
\hline

Sl No & Procedures & Expected result & Actual result & Pass or Fail\\  
\hline

1 & Server startup & Server starts  & Same as expected & Pass\\ 
\hline

2 & Client startup & Client starts & Same as expected & Pass\\
\hline

\end{tabular}}
\caption{Unit test cases and results}
\end{table}

