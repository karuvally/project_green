\chapter{Requirement Analysis}

\section{Purpose}
\par
The purpose of this system is to build a configuration management and monitoring
system that is dead simple to use. Highly efficient and feature rich
configuration management systems such as Ansible exists. But the problem is that
they have very steep learning curve. This is also the case with monitoring
systems such as Nagios which require considerable amount of setup from the part
of admins.\\

The learning curve makes these systems unsuitable to small office environments,
schools or colleges where the systems are managed by regular employees. NetDog
is dead simple, with zero learning curve. As long as a person is aware of the
command he/she wishes to execute on the remote machine, the person is ready to
use NetDog.\\

\section{Overall Description}
\par
NetDog has a client server architecture. The server is responsible for issuing
commands to the clients and is to be used by the administrator of the network.
The client application is to be run on machines on the network to be
administered.\\

After installing NetDog server or client on a machine, a unique public-private
key pair for the machine is generated which is then used for uniquely
identifying the machine and securing data transmission between the client and
server.\\

Once the server program is up and running, it listens on the port 1337 for
connections from clients. Once the client program is up, it starts listening on
port 1994. These ports serve dual purpose of facilitating communication and
allowing the identification of server and clients from the rest of the machines
on the network. Both NetDog server and client are daemons. They are system 
services which remain in memory and automatically starts during system boot.\\

When a client starts for the first time, it looks for active servers on the
network. When it finds one, it starts the pairing procedure. During the pairing
process, the client sends its hostname and public-key. The server in turn
provides the client with it's public key. These keys are then used for
identification and encrypted communication between machines.\\

The client and server uses public key encryption to identify and secure the
communication between them. The network admininstrator can issue commands from
the server machine which will then be sent to all the clients on the network.
A client can also contact the server occasionally, for example if the client
detects undseriable network traffic or if the system is overheating.\\

NetDog is capable of shutting down all the clients on the network at once by
remotely executing the shutdown command. It is also able to power up systems
which support remote Wake-On-LAN feature, by sending magic packets.\\

\subsection{Product Functions}
\begin{itemize}
    \item Bring all computers on the network up and down remotely
    \item Execute commands/scripts remotely on machines
    \item Copy files to remote machines without third party protocols
    \item Track and identify clients through IP changes
    \item Secure client-server communication using public key encryption
    \item List all machines on the network which are not clients (detect intruders)
    \item Centralized logging of all data regarding clients
    \item Web interface
\end{itemize}

\subsection{Hardware Requirements}
\begin{itemize}
    \item Intel Pentium IV or equivalent CPU
    \item 512 MB or more RAM
    \item 100 mbps Network Interface Card
\end{itemize}

\subsection{Software Requirements}
\begin{itemize}
    \item Linux
    \item Python 3
    \item pip
\end{itemize}

\section{Functional Requirements}
The system is should be designed to accept communication requests from many
clients at once. For this, a multi threaded server is necessary. Also, network
outages can occur during the operation. The server should be resilient enough
so that, it checks the status of the connection every once in a while and
restarts the communication process once the network is up and running again.
Also, the functions should extensively log themselves so that in case of a
system failure, the culprit can be easily found.

\section{Performance Requirements}
The system would need a gigabit ethernet controller for the server to make sure
that it can properly handle connections from multiple clients on the network.
The client can work satisfactorily well even on old hardware such as a 10 mbit
network card. There are no specific requirements for the CPU or the rest of the
hardware. The machine must be powerful enough to run a recent version of Linux,
which means any usable computing hardware would do.

