\chapter{Design And Implementation}

\section{Overall Design}
\par
NetDog follows the client-server architecture. What this means is that, NetDog
has both a client part and a server part. The server part is meant to be run
on the machine used by the system administrator, while the client part runs
on the rest of the machines on the network.\\

Both the server and client listens for connections from each other, allowing
the server to execute commands on clients and the client to report back any
progress. But the server has an additional capability that the clients lack, an
easy to use web interface that can be used by the administrator to control the
machines and configure the server.

\section{System Design}
\par
On a lower level, the design of NetDog is a bit different. NetDog is splitted
into three modules, the server module, the client module and the web server
module. The three of them has distinct functions. But the interesting fact is,
both the client module and the server module share the same underpinnings. All
the functions they use are from a library called libgreen, that was built
specifically for use in NetDog.

% debug: include image

\subsection{NetDog Server}
\par
The NetDog server module acts as the server. The server module resides in the
file ``netdog_server.py'' It uses functions part of the libgreen library to
setup a server working on the netdog protocol.\\

It listens for connections from clients on port 1337 and also provides the web
interface. The server is completeley multithreaded and this allows it to
perform tasks involving hundreds of machines at a surprisingly quick rate.

\subsection{NetDog client}
\par
The NetDog client module is the part of NetDog that is run on the client
machines. The client module listens for connections from the server at port
1994. Once a connection is received, it is handled as per the command it
contains.

\subsection{DataFlow Diagram}

