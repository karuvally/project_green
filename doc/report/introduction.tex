\chapter{Introduction}
	
\par
The netdog project started with the aim of designing a program which can easily
bring computers up and down remotely. The idea came from the realization that,
quite a few computers were left powered up when the college lab closes for the
day. Thus the initial name "Project Green".\\

A number of features were added to the feature list, most importantly tracking
computers even if their IP addresses changed, the ability to execute commands
or scripts on a specified range of machines with a single command, and the
ability to copy files to a range of machines without requiring protocols such
as SFTP or FTP.\\

The abilities does not end there and netdog provides many notable features such
as early warning of HDD failure on the machines on the network. The exhaustive
list of features is listed in later section. Once completed, netdog will be
a completely extensible system to which features an be easily added through
plugins.\\
\newpage 


The proposed real time traffic analyzer for road safety includes, tracking, and counting the vehicles using Blob Detection methods. In this project, first, we differentiate the foreground from background in frames by learning the background. Here, foreground detector detects the object and a binary computation is done to define rectangular regions around every detected object. To detect the moving object correctly and to remove the noise some morphological operations have been applied. Then the final counting is done by tracking the detected objects and their regions. Finally, vehicles in a predefined virtual detection zone are recorded and counted. The traffic details and counts are continuously updated in the server using a timer and users can register in an Android app to get notified about traffic in real time. Thus, the proposed system consists of three main modules. The first module constitutes the Android application for user registration and to retrieve notifications regarding traffic in real time. The second module deals with the vehicle detecting and counting and the count is updated to the server with the help of a timer. The third module is the server, which stores user details, traffic details, routes and traffic updates. 

