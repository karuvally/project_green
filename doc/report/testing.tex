\chapter{Testing and Implementation}
\section{All the possible testing methods done for the project}
\par System testing is the stage of implementation which is aimed at ensuring that the system works accurately and efficiently before live operation commences. Testing is the process of executing the program with the intent of finding errors and missing operations and also complete verification to determine whether the objective are met and the user requirements are satisfied.The ultimate aim is quality assurance.
Tests are carried and the results are compared with the expected document.In that case of erroneous results,debugging is done. Using detailed testing strategies a test plan is carried out on each module.
The test plan defines the unit,integration and system testing approach. The test scope includes the following:
A primary objective of testing application systems is to :assure that the system meets the full functional requirements, including quality requirements(Non functional requirements). At the end of the project development cycle, the user should find that the project has met or exceeded all of their expectations as detailed in requirements.
Any changes, additions or deletions to the requirements document,functional specification or design specification will be documented and tested at the highest level of quality allowed within the remaining time of the project and within the ability of the test team.
The secondary objective of testing application systems will be do:identify and expose all issues and associated risks, communicate all known issues are addressed in an appropriate matter before release
This test approach document describes the appropriate strategies, process, work flows and methodologies used to plan, organize, execute and manage testing of software project "Real-time Traffic Congestion Analyzer for Road Safety"
\newpage


\subsection{Unit Testing}
\textbf{Text Cases and Result}\\[0.5 cm]
\begin{table}[ht]

\resizebox{\textwidth}{!}{\begin{tabular}{|c| p{2cm} |p{3.7cm} |p{2.8cm}|c|}
\hline
 Sl No & Procedures & Expected result & Actual result & Pass or Fail \\  
 \hline \hline
 1 & Login into the system & Invalid login is Blocked & Same as expected & Pass \\ 
 \hline
 2 & Register the user & User is registered with valid username and password & Same as expected & Pass \\
 \hline
 3 & Video stream is inputted & Each frame is segmented into foreground and background objects & Same as expected & Pass\\
 \hline
4 & Region of interest is specified   & Foreground pixels from the background pixels are separated based on the intensity & Same as expected & Pass\\
\hline
5 & Image with ROI is  inputted & Frames are detected and vehicles are counted & Some false detection occurred & pass \\
\hline
 
\end{tabular}}
\caption{Unit test cases and results}
\end{table}


\subsection{Integration Testing}
\textbf{Text Cases and Result}\\[0.5 cm]
\begin{table}[ht]
\resizebox{\textwidth}{!}{\begin{tabular}{|c| p{2cm} |p{3.7cm} |p{2.8cm}|c|} 
 \hline
 Sl No & Procedures & Expected result & Actual result & Pass or Fail \\  
 \hline
 1 & User enters route details & Server is updated with traffic details & Same as expected & Pass \\ 
 \hline
 2 & Vehicle count is passed to the server &Count is updated every 15 minutes  & Same as expected & Pass\\
 \hline
 3 & Sends Notification & User receives traffic updates notification & Same as expected & Pass\\
 \hline
 
\end{tabular}}
\caption{Integration cases and result}
\end{table}

\subsection{System Testing}
\textbf{Text Cases and Result}\\[0.5 cm]
\begin{table}[ht]
\resizebox{\textwidth}{!}{\begin{tabular}{|c| p{2cm} |p{3.7cm} |p{2.8cm}|c|} 
 \hline
 Sl No & Procedures & Expected result & Actual result & Pass or Fail \\  
 \hline
 1 & Generation of traffic and route details & Generation as per the user input &  Same as expected & Pass \\
 \hline
 2 & Generation of Vehicle count &Depending upon the video stream  & Same as expected & Pass\\
 \hline
 3 & Generation of Notification & Notify user when heavy traffic occurs & Same as expected & Pass\\
 \hline

\end{tabular}}
\caption{System test cases and results}
\end{table}
\newpage


\section{Advantages and Limitations}

\par The proposed system consists of several advantages compared with previous systems. The vehicle detections can be traditionally achieved through inductive loop detector, infrared detector, radar detector or video-based solution. Compared to other techniques, the video-based solutions based on surveillance camera mounted outdoor are easily influenced by environments such as weather, illumination, shadow, etc. However, because video-based systems can offer several advantages over other methods such as traffic flow undisturbed, easily installed, conveniently modified, etc., they have drawn significant attention from researchers in the past decade. To overcome the previous limitations in vehicle detection and tracking, we present an improved method in this project to accurately separate the vehicle foreground from the adaptive background model with the help of pixel based adaptive segmenter.
\\
\\
\textbf{Advantages}\\ 
\begin{itemize}
\item Detects and count the vehicles 
\item Predicts Traffic flows 
\item Can be used to predict accident zone areas 
\item Ensures road safety
\item Notify registered users about traffic jam in real-time

\end{itemize}

\par There are also some limitations to the proposed system. Automatic vehicle counting system counts somewhat less than the actual number of vehicle due to congestion and heavy traffic flow situation in one scenario. Statistical computer vision method counts more numbers of vehicles than the actual number of vehicle in video sequences due to the false positive error factor. Table below shows the experimental results obtained by the proposed method and the comparison done with the similar purpose method.\\ \\
\newpage
\textbf{Experimental Results}\\[0.5 cm]
\begin{table}[ht]
\resizebox{\textwidth}{!}{\begin{tabular}{|c| p{2cm} |p{3.7cm} |p{2.8cm}|c|} 
 \hline
 Vehicle Video & Exact no. of vehicles in video & Number of vehicles calculated by the system & Success rate in percent \\  
 \hline
 1 & 17 & 13 & 76.47\% \\
 \hline
 2 & 27 & 24  & 88.88\% \\
 \hline
 3 & 59 & 57 & 96.67 \% \\
 \hline
Average & 103 & 94 & 91.26 \% \\
\hline
\end{tabular}}
\caption{Experimental Results}
\end{table}
\\
\\
\section{Future Extensions if possible}

The proposed system can be extended. From the data that server collects from user and OpenCV application, it can learn the properties of traffic, the way or route that the user mostly chooses and all. Thus, the extended system can predict the traffic hike in peak hours for each specific location. Also, users can get notified using google cloud messaging.
